\documentclass[a4paper,11ppt]{article}
\pagestyle{plain}
\usepackage{amsmath}
\usepackage{amssymb}
% We use \therefore from this package
\usepackage[retainorgcmds]{IEEEtrantools}
\begin{document}

\title{Assignment 4 - Differential Equations}
\date{\today}
\author{Mohan S Nayaka}
\renewcommand{\arraystretch}{1.5}

\textbf{1. }$\mathbf{y - x + xycotx + xy' = 0}$\\
\[y - x + xycotx + xy' = 0\]
\[\therefore x\frac{dy}{dx} + y + xycotx = x\]
\begin{equation}
\therefore \frac{dy}{dx} + y\left(\frac{1}{x}+cotx\right) = x \label{lineq}
\end{equation}
This is a linear first order differential equation where\\
$P(x) = \left(\frac{1}{x}+cotx\right)$ and $Q(x) = 1$\\
Thus, the integrating factor is $e^{\int{P(x)}dx}$
\begin{IEEEeqnarray*}{rCl}
\int{P(x)}dx & = & \int\left(\frac{1}{x} + cotx\right)dx\\
          & = & lnx+ln(sinx)\\
          & = & ln(xsinx)\\
\end{IEEEeqnarray*}
Thus, the integrating factor is\\
\begin{IEEEeqnarray*}{rCl}
e^{\int{P(x)}}dx &=& e^{ln(xsinx)}\\
                &=& xsinx\\
\end{IEEEeqnarray*}
Multiplying \eqref{lineq} by the integrating factor, we get,\\
\begin{IEEEeqnarray*}{rCl}
xsinx\frac{dy}{dx} + xysinx\left(\frac{1}{x}+cotx\right) &=& x sinx\\
\text{
Recognising the left hand-side as a derivative,
}\\
\frac{d}{dx}(xysinx) &=& xsinx
\end{IEEEeqnarray*}
\[\therefore xysinx = \int{xsinx}dx\]
\[\therefore xysinx = x\left(\int{sinx}\right)dx - \int\left({\frac{d}{dx}(x)\int{sinx}dx}\right)dx\]
\[\therefore xysinx = xcosx - \int{(1)(-cosx)}dx\]
Hence, the solution is :\\
\begin{center}
\begin{tabular}{|c|}
\hline
$xysinx = -xcosx + sinx + C$\\
\hline
\end{tabular}
\end{center}
\textbf{2. }$\mathbf{y - xy' = y'y^2e^y}$\\
\begin{IEEEeqnarray*}{rCl}
y=\frac{dy}{dx}\left(x+y^2e^y\right)
\end{IEEEeqnarray*}
\[\therefore \frac{dx}{dy} = \frac{x+y^2e^y}{y}\]
\[\therefore \frac{dx}{dy} = \frac{x}{y}+ye^y\]
\begin{equation}
\therefore \frac{dx}{dy} - \frac{x}{y} = ye^y \label{lineq2}
\end{equation}
This is of the form $\frac{dx}{dy} = P(y)x = Q(y)$\\
and thus a linear differential equation in y.\\
The integrating factor for \eqref{lineq2} is\\
\[  e^{\int(P(y))}dy\]
\[= e^{\int(\frac{-1}{y})dy}\]
\[= e^{-lny}\]
\[= \frac{1}{y}\]
Multiplying \eqref{lineq2} by the integrating factor, we get:\\
\[\frac{1}{y}\frac{dx}{dy} - \frac{x}{y^2} = e^y\]
Recognising the derivative on the left-hand-side, we have:\\
\[\frac{d}{dy}\left(\frac{x}{y}\right) = e^y\]
Thus the solution is:
{\renewcommand{\arraystretch}{1.5}
\begin{center}
\begin{tabular}{|c|}
\hline
$\frac{x}{y} = e^y + C$\\
\hline
\end{tabular}
\end{center}}
\end{document}